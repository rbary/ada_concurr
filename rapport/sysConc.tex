\documentclass[11pt,a4paper]{article}
\usepackage[top=2cm,bottom=2.5cm,left=2.5cm,right=2.5cm]{geometry}
\usepackage[french]{babel}
\usepackage{graphicx}
\usepackage[utf8]{inputenc}
\usepackage{verbatim}
\usepackage{fancyhdr}
\usepackage{listings}
\lstset{
	basicstyle=\fontsize{8}{10}\selectfont\ttfamily
}
\fancyhead[L]{}
\fancyhead[R]{}
\renewcommand{\footrulewidth}{0.5pt}
\fancyfoot[L]{\textit{Système concurrents}}
\fancyfoot[R]{}
\title{\textbf{Rapport de TP : Systèmes concurrents}}
\author{Charles Follet, Roland Bary}
\date{\today}
\begin{document}
\maketitle
\begin{center}
\maketitle{\textbf{Université de Pau et des Pays de l'Adour}}
\begin{center}
\includegraphics[scale=0.3]{logoUppa.png}
\end{center}
\end{center}
\thispagestyle{empty}
\newpage
\pagestyle{fancy}
\renewcommand{\contentsname}{Sommaire}
\tableofcontents
\newpage
\section*{Introduction}

\part{Rendez-vous ADA}
L'objectif ici est de résoudre ç l'aide de l'outil RDV ADA les problèmes de:
\begin{itemize}
\item Producteur/Consommateur
\begin{itemize}
\item[•]1 Producteur/ 1Consommateur: tampon de taille N
\item[•]1 producteur/1 Consommateur avec un tampon de taille 1
\end{itemize}
\item Lecteurs/Rédacteurs
\begin{itemize}
\item[•] Priorité aux Lecteurs
\item[•] Priorité aux Rédacteurs
\item[•] Priorité égales
\end{itemize}
\end{itemize}
\section{Producteur/ Consommateur}
\subsection{1 Producteur/ 1 Consommateur: tampon de taille N}
\subsection{Producteur/ Consommateur avec un tampon de taille 1}
\section{Lecteurs/ Rédacteurs}
\subsection{Priorité aux Lecteurs}
\subsection{Priorité aux Rédacteurs}
\subsection{Priorités égales}
\newpage
\part{Objets/Types protégés ADA 95}
L'objectif de cette partie est de :
\begin{itemize}
\item Rappeler le principe des Objets/Types protégés en ADA 95 : définition, principe, sémantique, différence avec les packages ADA
\item Comparer les  Objets/types protégés a d'autres outils comme  les sémaphores, RDV ADA, : pouvoir d'expression, difficulté/facilité d'utilisation, etc..
\item  Implémenter à l'aide Objets/Types protégés:
\begin{itemize}
\item[•] Producteur/Consommateur
\item[•] Lecteurs/ Rédacteurs: sans priorité et  priorité aux lecteurs
\item[•] Le problème du Carrefour a sens giratoire  
\end{itemize}
\end{itemize}
%%Beaucoup plus simples que les moniteurs de hoare
\section{Rappel des principes des Objets/Types protégés en ADA 95}
\subsection{Définition}
\subsection{Principe}
\subsection{Sémantique}
\subsection{Différence avec les packages ADA}
\section{}
\section{Implémentation à l'aide des Objets/Types protégés}
\subsection{Producteur/ Consommateur}
\subsection{Lecteurs/ Rédacteurs: sans priorité et priorité en ADA 95}
\subsection{Le problème du carrefour à sens giratoire}

\end{document}